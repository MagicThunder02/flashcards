\documentclass[a4paper,twoside]{article} 
\usepackage[utf8]{inputenc} 
\usepackage[italian]{babel} 
\usepackage{tikz} 
\usepackage{geometry} 
\geometry{ 
  a4paper, 
  total={170mm,257mm}, 
  left=10mm, 
  right=10mm, 
  top=10mm, 
  bottom=10mm 
} 

\newcommand{\flashcard}[4]{ 
  \begin{tikzpicture} 
    \draw (0,0) rectangle (11,6); 
    \ifodd\value{page} 
      \node[font=\large, text width=10cm, align=center] at (5.5,3) {\textbf{#2}\\[0.5cm] #3 \\ \footnotesize{Pagina: #4}}; 
    \else 
      \node[font=\large] at (5.5,3) {\textbf{#1}}; 
    \fi 
  \end{tikzpicture} 
} 

\begin{document} 
\pagestyle{empty} 

  \begin{center} 
    \vspace*{5cm} 
    \begin{tikzpicture}[remember picture, overlay, shift={(0,0)}] 
      \node at (-5.5,4) {\flashcard{AEAD}{Authenticated Encryption with Associated Data}{Crittografia autenticata con dati associati. Un algoritmo che fornisce sia riservatezza che integrità/autenticazione dei dati.}{88}};
      \node at (5.5,4) {\flashcard{AES}{Advanced Encryption Standard}{Standard di crittografia avanzata. Un algoritmo di crittografia a blocchi simmetrico ampiamente utilizzato.}{7}};
      \node at (-5.5,-2) {\flashcard{AH}{Authentication Header}{Intestazione di autenticazione. Un protocollo di sicurezza IPsec che fornisce autenticazione e integrità dei dati.}{184}};
      \node at (5.5,-2) {\flashcard{CA}{Certification Authority}{Autorità di certificazione. Un'entità attendibile che emette certificati digitali.}{94}};
      \node at (-5.5,-8) {\flashcard{CBC}{Cipher Block Chaining}{Cifratura a blocchi concatenati. Una modalità di funzionamento per algoritmi di crittografia a blocchi.}{60}};
      \node at (5.5,-8) {\flashcard{C.I.A.}{Confidentiality Integrity and Availability}{Riservatezza Integrità e Disponibilità. I tre principi fondamentali della sicurezza delle informazioni.}{36}};
      \node at (-5.5,-14) {\flashcard{CRL}{Certificate Revocation List}{Lista di certificati revocati. Un elenco di certificati digitali che non sono più validi.}{96}};
      \node at (5.5,-14) {\flashcard{CSP}{Credential Service Provider}{Fornitore di servizi di credenziali. Un componente che gestisce le credenziali degli utenti.}{107}};
      \node at (-5.5,-20) {\flashcard{CTS}{CipherText Stealing}{Furto di testo cifrato. Una tecnica di padding che evita l'aumento delle dimensioni del testo cifrato.}{63}};
      \node at (5.5,-20) {\flashcard{CTR}{Counter mode}{Modalità contatore. Una modalità di funzionamento per algoritmi di crittografia a blocchi.}{66}};
    \end{tikzpicture} 
  \end{center} 
  \newpage 

  \begin{center} 
    \vspace*{5cm} 
    \begin{tikzpicture}[remember picture, overlay, shift={(0,0)}] 
      \node at (-5.5,4) {\flashcard{AEAD}{Authenticated Encryption with Associated Data}{Crittografia autenticata con dati associati. Un algoritmo che fornisce sia riservatezza che integrità/autenticazione dei dati.}{88}};
      \node at (5.5,4) {\flashcard{AES}{Advanced Encryption Standard}{Standard di crittografia avanzata. Un algoritmo di crittografia a blocchi simmetrico ampiamente utilizzato.}{7}};
      \node at (-5.5,-2) {\flashcard{AH}{Authentication Header}{Intestazione di autenticazione. Un protocollo di sicurezza IPsec che fornisce autenticazione e integrità dei dati.}{184}};
      \node at (5.5,-2) {\flashcard{CA}{Certification Authority}{Autorità di certificazione. Un'entità attendibile che emette certificati digitali.}{94}};
      \node at (-5.5,-8) {\flashcard{CBC}{Cipher Block Chaining}{Cifratura a blocchi concatenati. Una modalità di funzionamento per algoritmi di crittografia a blocchi.}{60}};
      \node at (5.5,-8) {\flashcard{C.I.A.}{Confidentiality Integrity and Availability}{Riservatezza Integrità e Disponibilità. I tre principi fondamentali della sicurezza delle informazioni.}{36}};
      \node at (-5.5,-14) {\flashcard{CRL}{Certificate Revocation List}{Lista di certificati revocati. Un elenco di certificati digitali che non sono più validi.}{96}};
      \node at (5.5,-14) {\flashcard{CSP}{Credential Service Provider}{Fornitore di servizi di credenziali. Un componente che gestisce le credenziali degli utenti.}{107}};
      \node at (-5.5,-20) {\flashcard{CTS}{CipherText Stealing}{Furto di testo cifrato. Una tecnica di padding che evita l'aumento delle dimensioni del testo cifrato.}{63}};
      \node at (5.5,-20) {\flashcard{CTR}{Counter mode}{Modalità contatore. Una modalità di funzionamento per algoritmi di crittografia a blocchi.}{66}};
    \end{tikzpicture} 
  \end{center} 
  \newpage 

  \begin{center} 
    \vspace*{5cm} 
    \begin{tikzpicture}[remember picture, overlay, shift={(0,0)}] 
      \node at (-5.5,4) {\flashcard{DES}{Data Encryption Standard}{Standard di crittografia dei dati. Un algoritmo di crittografia a blocchi simmetrico ora considerato obsoleto.}{7}};
      \node at (5.5,4) {\flashcard{ECB}{Electronic Code Book}{Un metodo di crittografia a blocchi. Sconsigliato per messaggi lunghi a causa di vulnerabilità.}{59}};
      \node at (-5.5,-2) {\flashcard{ESP}{Encapsulating Security Payload}{Payload di sicurezza incapsulato. Un protocollo di sicurezza IPsec che fornisce riservatezza integrità e autenticazione dei dati.}{189}};
      \node at (5.5,-2) {\flashcard{GCM}{Galois/Counter Mode}{Modalità Galois/Contatore. Una modalità di funzionamento per algoritmi di crittografia a blocchi.}{90}};
      \node at (-5.5,-8) {\flashcard{HMAC}{Hash-based Message Authentication Code}{Codice di autenticazione del messaggio basato su hash. Un algoritmo per fornire integrità e autenticazione dei dati.}{85}};
      \node at (5.5,-8) {\flashcard{HPKP}{HTTP Public Key Pinning}{Fissaggio della chiave pubblica HTTP. Un meccanismo di sicurezza per impedire attacchi man-in-the-middle.}{224}};
      \node at (-5.5,-14) {\flashcard{HSTS}{HTTP Strict Transport Security}{Sicurezza del trasporto HTTP rigorosa. Un meccanismo di sicurezza per forzare l'utilizzo di HTTPS.}{223}};
      \node at (5.5,-14) {\flashcard{IGE}{Infinite Garble Extension}{Estensione infinita di Garble. Un algoritmo di crittografia non raccomandato ma spiegato a scopo didattico.}{88}};
      \node at (-5.5,-20) {\flashcard{IV}{Initialization Vector}{Vettore di inizializzazione. Un valore casuale utilizzato in alcuni algoritmi di crittografia.}{60}};
      \node at (5.5,-20) {\flashcard{Kc}{Connection Key}{Chiave di connessione. Una chiave utilizzata per crittografare le comunicazioni GSM.}{116}};
    \end{tikzpicture} 
  \end{center} 
  \newpage 

  \begin{center} 
    \vspace*{5cm} 
    \begin{tikzpicture}[remember picture, overlay, shift={(0,0)}] 
      \node at (-5.5,4) {\flashcard{DES}{Data Encryption Standard}{Standard di crittografia dei dati. Un algoritmo di crittografia a blocchi simmetrico ora considerato obsoleto.}{7}};
      \node at (5.5,4) {\flashcard{ECB}{Electronic Code Book}{Un metodo di crittografia a blocchi. Sconsigliato per messaggi lunghi a causa di vulnerabilità.}{59}};
      \node at (-5.5,-2) {\flashcard{ESP}{Encapsulating Security Payload}{Payload di sicurezza incapsulato. Un protocollo di sicurezza IPsec che fornisce riservatezza integrità e autenticazione dei dati.}{189}};
      \node at (5.5,-2) {\flashcard{GCM}{Galois/Counter Mode}{Modalità Galois/Contatore. Una modalità di funzionamento per algoritmi di crittografia a blocchi.}{90}};
      \node at (-5.5,-8) {\flashcard{HMAC}{Hash-based Message Authentication Code}{Codice di autenticazione del messaggio basato su hash. Un algoritmo per fornire integrità e autenticazione dei dati.}{85}};
      \node at (5.5,-8) {\flashcard{HPKP}{HTTP Public Key Pinning}{Fissaggio della chiave pubblica HTTP. Un meccanismo di sicurezza per impedire attacchi man-in-the-middle.}{224}};
      \node at (-5.5,-14) {\flashcard{HSTS}{HTTP Strict Transport Security}{Sicurezza del trasporto HTTP rigorosa. Un meccanismo di sicurezza per forzare l'utilizzo di HTTPS.}{223}};
      \node at (5.5,-14) {\flashcard{IGE}{Infinite Garble Extension}{Estensione infinita di Garble. Un algoritmo di crittografia non raccomandato ma spiegato a scopo didattico.}{88}};
      \node at (-5.5,-20) {\flashcard{IV}{Initialization Vector}{Vettore di inizializzazione. Un valore casuale utilizzato in alcuni algoritmi di crittografia.}{60}};
      \node at (5.5,-20) {\flashcard{Kc}{Connection Key}{Chiave di connessione. Una chiave utilizzata per crittografare le comunicazioni GSM.}{116}};
    \end{tikzpicture} 
  \end{center} 
  \newpage 

  \begin{center} 
    \vspace*{5cm} 
    \begin{tikzpicture}[remember picture, overlay, shift={(0,0)}] 
      \node at (-5.5,4) {\flashcard{Ki}{Subscriber authentication key}{Chiave di autenticazione dell'abbonato. Una chiave utilizzata per autenticare gli utenti GSM.}{116}};
      \node at (5.5,4) {\flashcard{LWC}{Lightweight Crypto}{Crittografia leggera. Algoritmi di crittografia progettati per dispositivi con risorse limitate.}{91}};
      \node at (-5.5,-2) {\flashcard{MAC}{Message Authentication Code}{Codice di autenticazione del messaggio. Un valore utilizzato per verificare l'integrità e l'autenticità di un messaggio.}{83}};
      \node at (5.5,-2) {\flashcard{MITM}{Man-in-the-Middle}{Uomo nel mezzo. Un attacco in cui un utente malintenzionato intercetta e altera le comunicazioni tra due parti.}{48}};
      \node at (-5.5,-8) {\flashcard{MNO}{Mobile Network Operator}{Operatore di rete mobile. Un fornitore di servizi di comunicazione mobile.}{114}};
      \node at (5.5,-8) {\flashcard{NAS}{Network Access Server}{Server di accesso alla rete. Un dispositivo che fornisce accesso alla rete agli utenti.}{147}};
      \node at (-5.5,-14) {\flashcard{NIST}{National Institute of Standards & Technology}{Istituto nazionale di standard e tecnologia. Un'agenzia governativa degli Stati Uniti che sviluppa standard tecnici.}{23}};
      \node at (5.5,-14) {\flashcard{NONCE}{Number used ONCE}{Numero utilizzato una sola volta. Un numero casuale o pseudo-casuale utilizzato in protocolli di sicurezza.}{60}};
      \node at (-5.5,-20) {\flashcard{OCSP}{Online Certificate Status Protocol}{Protocollo di stato del certificato online. Un protocollo per verificare la revoca del certificato.}{288}};
      \node at (5.5,-20) {\flashcard{OTP}{One-Time Password}{Password monouso. Una password valida per un singolo accesso.}{117}};
    \end{tikzpicture} 
  \end{center} 
  \newpage 

  \begin{center} 
    \vspace*{5cm} 
    \begin{tikzpicture}[remember picture, overlay, shift={(0,0)}] 
      \node at (-5.5,4) {\flashcard{Ki}{Subscriber authentication key}{Chiave di autenticazione dell'abbonato. Una chiave utilizzata per autenticare gli utenti GSM.}{116}};
      \node at (5.5,4) {\flashcard{LWC}{Lightweight Crypto}{Crittografia leggera. Algoritmi di crittografia progettati per dispositivi con risorse limitate.}{91}};
      \node at (-5.5,-2) {\flashcard{MAC}{Message Authentication Code}{Codice di autenticazione del messaggio. Un valore utilizzato per verificare l'integrità e l'autenticità di un messaggio.}{83}};
      \node at (5.5,-2) {\flashcard{MITM}{Man-in-the-Middle}{Uomo nel mezzo. Un attacco in cui un utente malintenzionato intercetta e altera le comunicazioni tra due parti.}{48}};
      \node at (-5.5,-8) {\flashcard{MNO}{Mobile Network Operator}{Operatore di rete mobile. Un fornitore di servizi di comunicazione mobile.}{114}};
      \node at (5.5,-8) {\flashcard{NAS}{Network Access Server}{Server di accesso alla rete. Un dispositivo che fornisce accesso alla rete agli utenti.}{147}};
      \node at (-5.5,-14) {\flashcard{NIST}{National Institute of Standards & Technology}{Istituto nazionale di standard e tecnologia. Un'agenzia governativa degli Stati Uniti che sviluppa standard tecnici.}{23}};
      \node at (5.5,-14) {\flashcard{NONCE}{Number used ONCE}{Numero utilizzato una sola volta. Un numero casuale o pseudo-casuale utilizzato in protocolli di sicurezza.}{60}};
      \node at (-5.5,-20) {\flashcard{OCSP}{Online Certificate Status Protocol}{Protocollo di stato del certificato online. Un protocollo per verificare la revoca del certificato.}{288}};
      \node at (5.5,-20) {\flashcard{OTP}{One-Time Password}{Password monouso. Una password valida per un singolo accesso.}{117}};
    \end{tikzpicture} 
  \end{center} 
  \newpage 

  \begin{center} 
    \vspace*{5cm} 
    \begin{tikzpicture}[remember picture, overlay, shift={(0,0)}] 
      \node at (-5.5,4) {\flashcard{OOB}{Out-Of-Band}{Fuori banda. La trasmissione di informazioni tramite un canale diverso dal canale principale.}{69}};
      \node at (5.5,4) {\flashcard{PCI DSS}{Payment Card Industry Data Security Standard}{Standard di sicurezza dei dati del settore delle carte di pagamento. Un insieme di requisiti di sicurezza per le organizzazioni che gestiscono i dati delle carte di pagamento.}{111}};
      \node at (-5.5,-2) {\flashcard{PF}{Packet Filter}{Filtro di pacchetti. Un tipo di firewall che filtra i pacchetti di rete in base a regole predefinite.}{173}};
      \node at (5.5,-2) {\flashcard{PKI}{Public-Key Infrastructure}{Infrastruttura a chiave pubblica. Un sistema per la gestione delle chiavi pubbliche e dei certificati digitali.}{96}};
      \node at (-5.5,-8) {\flashcard{RADIUS}{Remote Authentication Dial-In User Service}{Servizio di autenticazione remota per utenti dial-in. Un protocollo di rete per l'autenticazione l'autorizzazione e la contabilità.}{151}};
      \node at (5.5,-8) {\flashcard{RFC}{Request for Comments}{Richiesta di commenti. Una serie di documenti che descrivono standard Internet e protocolli correlati.}{88}};
      \node at (-5.5,-14) {\flashcard{RSA}{Rivest-Shamir-Adleman}{Un algoritmo di crittografia asimmetrica ampiamente utilizzato.}{91}};
      \node at (5.5,-14) {\flashcard{S/KEY}{Bellcore's S/Key One-Time Password System}{Un sistema di password monouso basato su hash.}{119}};
      \node at (-5.5,-20) {\flashcard{SAD}{Security Association Database}{Database delle associazioni di sicurezza. Un database che contiene informazioni sulle associazioni di sicurezza IPsec.}{181}};
      \node at (5.5,-20) {\flashcard{SHA}{Secure Hash Algorithm}{Algoritmo di hash sicuro. Una famiglia di funzioni di hash crittografiche.}{88}};
    \end{tikzpicture} 
  \end{center} 
  \newpage 

  \begin{center} 
    \vspace*{5cm} 
    \begin{tikzpicture}[remember picture, overlay, shift={(0,0)}] 
      \node at (-5.5,4) {\flashcard{OOB}{Out-Of-Band}{Fuori banda. La trasmissione di informazioni tramite un canale diverso dal canale principale.}{69}};
      \node at (5.5,4) {\flashcard{PCI DSS}{Payment Card Industry Data Security Standard}{Standard di sicurezza dei dati del settore delle carte di pagamento. Un insieme di requisiti di sicurezza per le organizzazioni che gestiscono i dati delle carte di pagamento.}{111}};
      \node at (-5.5,-2) {\flashcard{PF}{Packet Filter}{Filtro di pacchetti. Un tipo di firewall che filtra i pacchetti di rete in base a regole predefinite.}{173}};
      \node at (5.5,-2) {\flashcard{PKI}{Public-Key Infrastructure}{Infrastruttura a chiave pubblica. Un sistema per la gestione delle chiavi pubbliche e dei certificati digitali.}{96}};
      \node at (-5.5,-8) {\flashcard{RADIUS}{Remote Authentication Dial-In User Service}{Servizio di autenticazione remota per utenti dial-in. Un protocollo di rete per l'autenticazione l'autorizzazione e la contabilità.}{151}};
      \node at (5.5,-8) {\flashcard{RFC}{Request for Comments}{Richiesta di commenti. Una serie di documenti che descrivono standard Internet e protocolli correlati.}{88}};
      \node at (-5.5,-14) {\flashcard{RSA}{Rivest-Shamir-Adleman}{Un algoritmo di crittografia asimmetrica ampiamente utilizzato.}{91}};
      \node at (5.5,-14) {\flashcard{S/KEY}{Bellcore's S/Key One-Time Password System}{Un sistema di password monouso basato su hash.}{119}};
      \node at (-5.5,-20) {\flashcard{SAD}{Security Association Database}{Database delle associazioni di sicurezza. Un database che contiene informazioni sulle associazioni di sicurezza IPsec.}{181}};
      \node at (5.5,-20) {\flashcard{SHA}{Secure Hash Algorithm}{Algoritmo di hash sicuro. Una famiglia di funzioni di hash crittografiche.}{88}};
    \end{tikzpicture} 
  \end{center} 
  \newpage 

  \begin{center} 
    \vspace*{5cm} 
    \begin{tikzpicture}[remember picture, overlay, shift={(0,0)}] 
      \node at (-5.5,4) {\flashcard{SIM}{Subscriber Identity Module}{Modulo di identità dell'abbonato. Una smart card utilizzata nei telefoni cellulari GSM.}{114}};
      \node at (5.5,4) {\flashcard{SRES}{Signed Response}{Risposta firmata. Un valore utilizzato nell'autenticazione GSM.}{115}};
      \node at (-5.5,-2) {\flashcard{SPI}{Security Parameter Index}{Indice dei parametri di sicurezza. Un valore utilizzato per identificare le associazioni di sicurezza IPsec.}{184}};
      \node at (5.5,-2) {\flashcard{SSO}{Single Sign-On}{Accesso singolo. Un sistema di autenticazione che consente agli utenti di accedere a più sistemi con un solo insieme di credenziali.}{131}};
      \node at (-5.5,-8) {\flashcard{SSL}{Secure Sockets Layer}{Livello socket sicuro. Un protocollo di sicurezza per la comunicazione su Internet ora sostituito da TLS.}{207}};
      \node at (5.5,-8) {\flashcard{TCP}{Transmission Control Protocol}{Protocollo di controllo della trasmissione. Un protocollo di trasporto affidabile utilizzato per la comunicazione su Internet.}{101}};
      \node at (-5.5,-14) {\flashcard{TLS}{Transport Layer Security}{Sicurezza del livello di trasporto. Un protocollo di sicurezza per la comunicazione su Internet.}{207}};
      \node at (5.5,-14) {\flashcard{TOTP}{Time-based One-Time Password}{Password monouso basata sul tempo. Un tipo di OTP che utilizza l'ora corrente per generare la password.}{120}};
      \node at (-5.5,-20) {\flashcard{UDP}{User Datagram Protocol}{Protocollo datagramma utente. Un protocollo di trasporto non affidabile utilizzato per la comunicazione su Internet.}{189}};
      \node at (5.5,-20) {\flashcard{VPN}{Virtual Private Network}{Rete privata virtuale. Una connessione di rete sicura che estende una rete privata su una rete pubblica.}{183}};
    \end{tikzpicture} 
  \end{center} 
  \newpage 

  \begin{center} 
    \vspace*{5cm} 
    \begin{tikzpicture}[remember picture, overlay, shift={(0,0)}] 
      \node at (-5.5,4) {\flashcard{SIM}{Subscriber Identity Module}{Modulo di identità dell'abbonato. Una smart card utilizzata nei telefoni cellulari GSM.}{114}};
      \node at (5.5,4) {\flashcard{SRES}{Signed Response}{Risposta firmata. Un valore utilizzato nell'autenticazione GSM.}{115}};
      \node at (-5.5,-2) {\flashcard{SPI}{Security Parameter Index}{Indice dei parametri di sicurezza. Un valore utilizzato per identificare le associazioni di sicurezza IPsec.}{184}};
      \node at (5.5,-2) {\flashcard{SSO}{Single Sign-On}{Accesso singolo. Un sistema di autenticazione che consente agli utenti di accedere a più sistemi con un solo insieme di credenziali.}{131}};
      \node at (-5.5,-8) {\flashcard{SSL}{Secure Sockets Layer}{Livello socket sicuro. Un protocollo di sicurezza per la comunicazione su Internet ora sostituito da TLS.}{207}};
      \node at (5.5,-8) {\flashcard{TCP}{Transmission Control Protocol}{Protocollo di controllo della trasmissione. Un protocollo di trasporto affidabile utilizzato per la comunicazione su Internet.}{101}};
      \node at (-5.5,-14) {\flashcard{TLS}{Transport Layer Security}{Sicurezza del livello di trasporto. Un protocollo di sicurezza per la comunicazione su Internet.}{207}};
      \node at (5.5,-14) {\flashcard{TOTP}{Time-based One-Time Password}{Password monouso basata sul tempo. Un tipo di OTP che utilizza l'ora corrente per generare la password.}{120}};
      \node at (-5.5,-20) {\flashcard{UDP}{User Datagram Protocol}{Protocollo datagramma utente. Un protocollo di trasporto non affidabile utilizzato per la comunicazione su Internet.}{189}};
      \node at (5.5,-20) {\flashcard{VPN}{Virtual Private Network}{Rete privata virtuale. Una connessione di rete sicura che estende una rete privata su una rete pubblica.}{183}};
    \end{tikzpicture} 
  \end{center} 
  \newpage 

\end{document}